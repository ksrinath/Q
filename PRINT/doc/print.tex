\documentclass[letterpaper]{article}
\input{../../latex/styles/ramesh_abbreviations}
\usepackage{times}
\usepackage{helvet}
\usepackage{courier}
\usepackage{hyperref}
\usepackage{fancyheadings}
\pagestyle{fancy}
\usepackage{pmc}
\usepackage{graphicx}
\setlength\textwidth{6.5in}
\setlength\textheight{8.5in}
\begin{document}
\title{Print to CSV}
\author{ Ramesh Subramonian }
\maketitle
\thispagestyle{fancy}
\lfoot{{\small Data Analytics Team}}
\cfoot{}
\rfoot{{\small \thepage}}

\section{Introduction}

This document describes how we print data in one or more vectors into a CSV
file. 

\subsection{Invocation}

\begin{verbatim}
print(<table of vectors|vector>, [filter], [destination])
\end{verbatim}
So, the first argument to print could be just \(x\) or 
it could be \(\{x, y\}\)

\subsection{Print Filters}

There are a few ways in which we can instruct the print to {\bf not}
print all the elements of the vector. We can use a range (Section~\ref{range})
or a ``where clause'' represented as a bit vector, Section~\ref{bit_vector}.

\subsubsection{Range}
\label{range}

Support
\verb+lb:ub+ where {\tt lb} is the lower bound (inclusive) and {\tt
ub} is the upper bound (exclusive). So a range of \(2:4\) would mean printing
out the \(2^{th}\) element and the \(3^{th}\) element. 
Recall that we do C style indexing, so the first element is the \(0^{th}\)
element. 
Some time in the future --- support Python's range specification style. 

\subsubsection{Bit Vector}
\label{bit_vector}
Prints only those rows where the corresponding bit is set.

\subsection{Print Destination}

The destination of a print can be either 
\be
\item stdout (user does not specify a file name)
\item a text file 
\ee

It is important to note that the output of print should be consumable by the
load csv function. For example, if the value to be printed contains a special
character --- comma, double quote, eoln or backslash --- then it must be
suitable escaped and possibly quoted as well. A null value will be printed out
as two consescutive dquote characters.

\section{Printing a single vector}
\label{single_vector}

Let's start with describing how a single vector should be printed. As
described in the data loading specification, every type that is
registered with Q should have a C function that (i) converts an ascii
string into the C type and (ii) converts a fixed set of bytes into an ascii
string. We simply iterate through the elements and apply the text converter to
each consecutive block. For example, we would apply {\t fprintf} with the
\verb+%lf+ format on every successive 8 bytes if the field type were
{\tt double}

\section{Printing multiple vectors}
\label{multiple_vectors}

We require that 
\be
\item 
the first argument is a Lua table where each element is a single vector. 
\item all vectors have the same length
\ee

\subsection{Extensions}
In this section, we describe what happens when the constraints imposed above are
selectively relaxed.
\subsubsection{Mismatched number of elements}
In this case, the vector with insufficient elements starts printing out null
values once it has exhausted all its values. So, if \(x = \{1, 2,3\}\) and \(y
= \{4, 5, 6, 7\}\), then we would get
\begin{verbatim}
1,4
2,5
3,6
"",7
\end{verbatim}

\subsubsection{Scalar not vector}
If the table looked like \(\{1, x\}\) where \(x = \{2, 3, 4 \}\), then we would
get
\begin{verbatim}
1,2
1,3
1,4
\end{verbatim}

\end{document}
