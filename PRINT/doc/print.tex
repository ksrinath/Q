\documentclass[letterpaper]{article}
\input{../../latex/styles/ramesh_abbreviations}
\usepackage{times}
\usepackage{helvet}
\usepackage{courier}
\usepackage{hyperref}
\usepackage{fancyheadings}
\pagestyle{fancy}
\usepackage{pmc}
\usepackage{graphicx}
\setlength\textwidth{6.5in}
\setlength\textheight{8.5in}
\begin{document}
\title{Print to CSV}
\author{ Ramesh Subramonian }
\maketitle
\thispagestyle{fancy}
\lfoot{{\small Data Analytics Team}}
\cfoot{}
\rfoot{{\small \thepage}}

\section{Introduction}

This document describes how we print data in one or more vectors into a CSV
file. 

\subsection{Invocation}

\begin{verbatim}
print(<table of vectors|vector>, [filter], [destination])
\end{verbatim}
So, the first argument to print could be just \(x\) or 
it could be \({x, y}\)}.

\subsection{Print Filters}

There are a few ways in which we can instruct the print to {\bf not}
print all the elements of the vector.

\subsubsection{Range}

Try and support Python's range specification style. Else, support
\verb+lb:ub+ where {\tt lb} is the lower bound (inclusive) and {\tt
  ub} is the upper bound (exclusive)

\subsubsection{Bit Vector}
Prints only those rows where the corresponding bit is set.

\subsection{Print Destination}

The destination of a print can be either 
\be
\item stdout (user does not specify a file name)
\item a text file 
\ee

\section{Printing a single vector}
\label{Single_Vector}

Let's start with describing how a single vector should be printed. As
described in the data loading specification, every type that is
registered with Q should have a C function that converts an ascii
string into the C type and vice versa.

\section{Printing multiple vectors}
\label{Single_Vector}

We require that 
\be
\item 
the first argument is a Lua table where each element is a single vector. 
\item 

\subsection{Extensions}
In this section, we describe what happens when the constraints imposed above are
selectively relaxed.
\subsubsection

